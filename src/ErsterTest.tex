% \documentclass[ngerman]{book}	%Dokumentenvorlage "Buch"
\documentclass[ngerman,parskip]{scrbook}

\usepackage[utf8]{inputenc} 	%ermöglicht die direkte Eingabe von Umlauten
\usepackage[T1]{fontenc} 		%Ausgabe aller zeichen in einer T1-Codierung (wichtig für die Ausgabe von Umlauten!)
\usepackage{babel} 				%deutsche Trennungsregeln und Übersetzung der festcodierten Überschriften
\usepackage{amsmath}			%Erforderlich für Formeln
\usepackage{graphicx}            %Erforderlich für Bilder
\usepackage[onehalfspacing]{setspace}

% \setlength{\parindent}{0ex}

%%%%%%%%%%%%%%%%%%%%%%%%%%%%%
\usepackage[backend=biber,style=alphabetic,natbib=true,hyperref=true]{biblatex}
\addbibresource{literatur1.bib} % Tectonic kann nicht mit bib Dateien arbeiten, die in Unterverzeichnissen liegen.
%%%%%%%%%%%%%%%%%%%%%%%%%%%%%

\begin{document}
\tableofcontents

\chapter{Grundlagen}

Glückwunsch, Sie haben Ihr erstes \LaTeX-Dokument erzeugt. Dieses Dokument
dient dazu, sich in das Arbeiten mit \LaTeX~einzuarbeiten. Sehen Sie es mir
bitte nach, wenn die Erklärungen in diesem Dokument an der Oberfläche bleiben.
Ziel dieses Dokuments ist es, einen schnellen Einstieg zu ermöglichen. Details
zu allen Befehlen finden Sie im Internet in den jeweiligen
Package-Dokumentationen oder in unzähligen Foren mit vielen hilfreichen
Code-Beispielen und erforderlichen Maßnahmen bei Fehlermeldungen.
%Anmerkung: Der Befehl \; bewirkt einen erzwungen Abstand hinter dem Befehl
%\LaTeX. Ohne \; folgt das Wort danach ansonsten ohne Zwischenraum. Eine
%Alternative zu \; wäre das Tilde-Symbol ~ . Dieses fügt aber ein geschütztes
%Leerzeichen ein, an dem kein Zeilenumbruch stattfinden soll. Neben \; gibt es
%auch Befehle mit leicht anderen Abständen, z.B. einen kleinen \, oder einen
%sehr großen \!

Die \LaTeX-Befehle, welche ich hier einführe, sind abwechselnd sowohl im
PDF-Dokument, als auch als auskommentierte Zeilen in der .tex-Datei zu sehen.
Auskommentierte Zeilen sind durch ein Prozentzeichen am Anfang der Zeile zu
erkennen. Durch Löschen des Prozentzeichens wird die Zeile entkommentiert und
wird zu kompilierbarem \LaTeX-Code. Gehen Sie dieses Dokument unbedingt in
einem \LaTeX-Editor durch, damit Sie immer die beiden Sichten, den \LaTeX-Code
und das erzeugte PDF-Dokument, nebeneinander sehen. Durch Rechtsklick sowohl an
einer Stelle des Textes in der .tex-Datei, als auch an einer Stelle des
PDF-Dokuments können Sie zur entsprechenden Stelle im PDF-Dokument, bzw. im
Quelltext springen. Das ist sehr hilfreich beim Navigieren.

Damit diese Kurzanleitung funktioniert, gehen Sie bitte genau nach der
angegebenen Reihenfolge vor und setzen die Vorschläge genau so um, wie
beschrieben. Behalten Sie jede Änderung bei, da sonst möglicherweise die
nachfolgenden Schritte nicht funktionieren könnten.

Los geht's: Sie finden es seltsam, dass die ersten Zeile eines Abschnitts
eingerückt ist? Ändern Sie das einfach, indem Sie den Befehl

\textit{$\backslash$setlength\{$\backslash$parindent\}\{0ex\}}
% Sie können hierzu die bereits vorhandene Zeile mit
% \setlength{\parindent}{0ex} in der Präambel entkommentieren.

in die Präambel schreiben. Als \glqq Präambel\grqq\; wird der Teil vor der
Zeile mit dem Befehl \textit{$\backslash$begin\{document\}} genannt.

Wenn Sie den Code dieser Vorlage ein wenig studieren, sehen Sie, dass mit dem Befehl

\textit{$\backslash$chapter\{Grundlagen\}}
%\chapter{Grundlagen}

die erste Kapitelüberschrift gesetzt wird. Mit dem gewählten Dokumentenstil,
welcher in der ersten Zeile mit \glqq book\grqq\; angegeben ist, erzeugt
\textit{$\backslash$chapter\{\}} eine formatiere Kapitelüberschrift. Im
Hintergrund hat \LaTeX bereits vermerkt, dass eine Überschrift in der ersten
Ebene mit dem Namen \glqq Grundlagen\grqq\; vorhanden ist, was zur Erstellung
des Inhaltsverzeichnisses genutzt werden kann.

Es gibt viele andere Dokumentenstile neben dem hier angegebenen \glqq
book\grqq. Für wissenschaftliche Dokumentationen im Rahmen Ihres Studiums
eignet sich am besten der Dokumentenstil \glqq scrbook\grqq\;. Ersetzen Sie in
der ersten Zeile die Angabe \glqq book\grqq\; mit \glqq scrbook\grqq.
%So sieht dann die erste Zeile aus: \documentclass[ngerman]{scrbook}

Die Absätze sind noch etwas dicht beieinander. Bauen Sie die Option \glqq
parskip\grqq\; in die erste Zeile ein. Optionen werden für gewöhnlich in
eckigen Klammern angegeben und mit Kommata von anderen Optionen getrennt.
%So sieht dann die erste zeile aus: \documentclass[ngerman,parskip]{book}

Wenn Sie den Zeilenabstand noch erhöhen wollen, dann schreiben Sie den nachfolgenden Befehl

\textit{$\backslash$usepackage[onehalfspacing]\{setspace\}}
%Sie können hierzu die bereits vorhandene Zeile mit \usepackage[onehalfspacing]{setspace} in der Präambel entkommentieren.

in die Präambel.

Die bisher vorgenommenen Einstellungen in der Präambel wirken auf das gesamte
Dokument. Ziel ist es, einen einheitlichen Stil im gesamten Dokument zu
erhalten. Sie können natürlich an einer speziellen Stelle des Dokuments
Änderungen des Stils erzwingen. Z.B. können manuelle Zeilenumbrüche mit den
Befehlen \textit{$\backslash$newline} oder $\backslash\backslash$ erzeugt
werden. Einen Seitenumbruch erzwingt man mit \textit{$\backslash$pagebreak}
oder \textit{$\backslash$clearpage}. Zusätzliche vertikale Abstände (z.B. von
2\;cm) können mit $\backslash$vspace\{2cm\} hinzugefügt werden. Vermeiden Sie
solche Vorgehensweisen unbedingt und greifen Sie nur in Ausnahmefällen darauf
zurück. Auch wenn Ihnen manches auf den ersten Blick seltsam erscheint,
vertrauen Sie dem Textsatzprogramm \LaTeX\; und nehmen erforderliche
Anpassungen an einzelnen Stellen erst ganz am Schluss vor, wenn alle Inhalte im
Dokument vorhanden sind. Diese Arbeitsweise hat außerdem den Vorteil, dass Sie
sich zunächst einmal auf den Inhalt konzentrieren.

\chapter{Gliederung und Listen}

Die oberste Ebene der Überschriften haben Sie schon kennen gelernt, diese wird
mit \textit{\textit{$\backslash$chapter\{\}}} angegeben. Darunter befindet sich
die Überschrift der zweiten Ebene \textit{\textit{$\backslash$section\{\}}} und
der dritten Ebene \textit{$\backslash$subsection\{\}}, usw. Entfernen Sie bitte
das Prozentzeichen in der nächsten Zeile.

\section{Gliederung}

Sie haben gesehen, dass sich die Überschrift automatisch formatiert und
nummeriert. Wenn Sie schon mal die unschöne Erfahrung mit einem anderen
Textverarbeitungsprogramm gemacht haben, dass bei großen Dokumenten die
Nummerierung nicht einwandfrei funktioniert, so bleiben Ihnen in \LaTeX\;
solche schlechten Erfahrungen erspart. Manchmal dauert es zwar, den richtigen
Befehl zu finden und richtig anzuwenden, wenn es aber mal funktioniert, dann
immer. Weil es so schön ist, entfernen die bitte gleich nochmals das
Prozentsymbol in der nächsten Zeile.

\section{Listen}

Es gibt einige unterschiedliche Listentypen, hier die zwei wichtigsten.
Markieren Sie zum Aktivieren jeweils alle Zeilen ab \textit{$\backslash$begin}
bis \textit{$\backslash$end} und benutzen Sie die Tastenkombination STRG-T
(bzw. CTRL-T) zum Entfernen des Prozentzeichens. Dieselbe Tastenkombination
kann auch zur Aktivierung der Kommentarfunktion benutzt werden.

\begin{itemize}
	\item Punkt 1
	\item Punkt 2
	\item Punkt 3
\end{itemize}

\begin{enumerate}
	\item kommt es anders,
	\item als man denkt.
\end{enumerate}

\chapter{Bilder und Tabellen}

Im nachfolgenden Beispiel wird ein Bild in das Dokument eingefügt, mit einer
Bildunterschrift versehen und das Bild erhält einen Kennzeichner (Label), um im
umgebenden Text auf die Abbildung referenzieren zu können. Das Bild wird
zentriert und wird in eine Gleitumgebung gepackt. Gleitumgebung bedeutet, dass
\LaTeX\; entscheidet, wo genau die Abbildung samt Abbildungsunterschrift
erscheinen soll. Die Option \glqq hbt\grqq\; in der ersten Zeile gibt die
Reihenfolge der Präferenz des Benutzers bzgl. der Platzierung an: Wenn möglich,
wird das Objekt an genau dieser Stelle sichtbar (\textbf{h}ere), wenn dies
nicht möglich ist, versucht \LaTeX\, das Objekt unten an der Seite
(\textbf{b}ottom) zu platzieren, wenn auch das aus irgendwelchen Gründen
ungünstig ist, dann setzt \LaTeX\, das Objekt oben auf die Seite
(\textbf{t}op). Um den folgenden Block zur Einbindung eines Bildes einbinden zu
können, muss in der Präambel das Package graphicx eingebunden werden.
%Sie können hierzu die bereits vorhandene Zeile mit \usepackage{graphicx} in
%der Präambel entkommentieren

\begin{figure}[hbt]
	\centering
	\includegraphics{images/testbild.png}
	\caption{Testbild}
	\label{fig:test_bild_01}
\end{figure}

Das Bild testbild.png ist in einem Unterordner mit dem Namen \glqq
images\grqq\;gespeichert. Achten Sie darauf, dass bei Pfadangaben kein
Backslash, sondern ein normaler Schrägstrich verwendet wird. Ein kleiner
Exkurs, auf den hier nicht näher eingegangen wird, welcher aber wichtig für
größere Dokumente ist: Üblicherweise werden nicht nur Bilder in Unterordnern
gespeichert, sondern jedes Kapitel eines Dokuments ist in einer eigenen
.tex-Datei in einem Unterordner, z.B. mit dem Namen chapter, gespeichert.
Eingebunden werden dann die einzelnen Kapitel über den Befehl
\textit{$\backslash$include\{chapter/datei.tex\}}. Das hat nicht nur den
Vorteil der besseren Übersicht, sondern es lässt sich so durch Auskommentieren
der Zeile mit dem include-Befehl ein ganzes Kapitel beim Kompilieren \glqq
ausblenden\grqq. Das spart Zeit beim Kompilieren, wenn gerade an einem anderen
Kapitel gearbeitet wird.

Tabellen besitzen eine andere Gleitumgebung \glqq table\grqq . Zu beachten ist,
dass Tabellen eine Überschrift haben, im Gegensatz zu Bildern, welche eine
Beschriftung unterhalb haben. 

\begin{table}[hbt]
	\centering
	\captionabove{Tabelle mit wenig Inhalt}
	\label{tab:test_tabelle_01}
	\begin{tabular}{ccc}
		\textbf{Spalte 1} & \textbf{Spalte 2} & \textbf{Spalte 3} \\
		\hline
		\hline
		Wert 1 & Wert 2 & Wert 3 \\
		\hline
	\end{tabular}
\end{table}

\chapter{Referenzen}

Referenzen auf Bilder und Tabellen können im Text leicht eingebunden werden.
Entfernen Sie das Prozentsysmbol am Anfang des nächsten Absatzes, um die
Beispiele zu sehen.

Abbildung~\ref{fig:test_bild_01} und Tabelle~\ref{tab:test_tabelle_01} können
im Text auf unterschiedliche Arten referenziert werden. Es kann auch ausgegeben
werden, dass sich Tabelle~\ref{tab:test_tabelle_01} auf Seite
\pageref{tab:test_tabelle_01} befindet. Referenzen können auch für Formeln,
Überschriften, etc. eingesetzt werden. \LaTeX-Editoren helfen bei der
Referenzierung durch eine Autovervollständigen-Funktion.

\chapter{Formeln}

\LaTeX\, zeigt zum Verfassen von Schriften mit vielen Formeln eine besondere Stärke. Kommentieren Sie den nachfolgenden Absatz aus, um Beispiele zu sehen.

Formel~\ref{eqn:newton} stellt das zweiten Newtonsche Axiom in allgemeiner Form dar:
\begin{equation}\label{eqn:newton}
\vec{F}= \dfrac{\text{d}\vec{p}}{\text{d}t}=\dfrac{\text{d}m}{\text{d}t}\vec{v}+\dfrac{\text{d}\vec{v}}{\text{d}t}m
\end{equation}


\chapter{Verzeichnisse}

Das wichtigste Verzeichnis einer wissenschaftlichen Arbeit ist das
Literaturverzeichnis. Hier ist ein wenig Geduld gefragt, denn es sind ein paar
Vorbereitungen dazu erforderlich. Es gibt einige verschiedene Möglichkeiten das
Literaturverzeichnis einzubinden. Hier ist eine Vorgehensweise gezeigt, welche
sich in eigenen Tests am robustesten erwiesen hat.

Zunächst einmal müssen in einer Textdatei mit Endung .bib die bibliografischen
Daten gespeichert sein. Ein Beispiel ist im Ordner \textit{literature} in der
Datei \textit{literature.bib} gegeben. Sie finden in der Datei die Daten des
Physik-Buchs von Paul Allen Tipler aus dem Jahr 2019. Natürlich können in der
.bib-Datei noch weitere Literatureinträge vorhanden sein, welche jeweils mit
dem {@}-Zeichen, dem Dokumententyp und einem Kürzel, hier \glqq
Tipler.2019\grqq , eingeleitet werden. Das Kürzel wird später im
\LaTeX-Dokument verwendet, um auf die Literaturquelle zu referenzieren.

Es wäre natürlich mühsam, die bibliografischen Daten von Hand im richtigen
Format zu schreiben. Daher ist die Verwendung eines
Literaturverwaltungsprogramms mit einer Export-Funktion für die .bib-Datei von
Vorteil. Beispiele sind JabRef oder Citavi. Probieren Sie die für Sie
geeignetste Methode aus und spielen Sie den Vorgang vom Ermitteln der
bibliografischen Daten bis zum Zitieren in \LaTeX\; komplett durch, um sich mit
den Gegebenheiten vertraut zu machen.

In der Präambel ist \LaTeX\; noch mitzuteilen, dass ein Literaturverzeichnis gedruckt werden soll. Entkommentieren Sie die beiden Zeilen in der Präambel mit den beiden Befehlen

\textit{$\backslash$usepackage[backend=biber,style=alphabetic,natbib=true,hyperref=true]\{biblatex\}}\\
\textit{$\backslash$addbibresource\{literature/literatur1.bib\}}
%Sie können hierzu die bereits vorhandenen Zeilen in der Präambel entkommentieren
%\usepackage[backend=biber,style=alphabetic,natbib=true,hyperref=true]{biblatex}
%\addbibresource{literature/literatur1.bib}

Nun muss noch der \LaTeX-Editor konfiguriert werden, denn beim Kompilieren muss
ein Hilfsprogramm \glqq biber\grqq\; benutzt werden, um das
Literaturverzeichnis zu erzeugen. Standardmäßig dürfte in den meisten Editoren
\glqq BibTeX\grqq\; vorkonfiguriert sein. Für die hier beschriebenen Befehle
funktioniert dieses Programm jedoch nicht. Darüber hinaus ist \glqq
BibTeX\grqq\; nicht der neueste Stand. In TeXstudio gehen Sie bitte im Menü in
Optionen-TeXstudio konfigurieren...-Erzeugen und stellen bei
Standard-Bibliographiprogramm \glqq biber\grqq\; ein.

Ein Literaturverzeichnis wird natürlich erst gedruckt, wenn auch etwas zitiert
wird. Mit dem Befehl \textit{$\backslash$autocite\{Tipler.2019\}} wird eine
Quelle zitiert und ins Verzeichnis aufgenommen. Entfernen Sie die
Prozentzeichen in den beiden nachfolgenden Zeilen.
Ein Literaturverweis auf das Physik-Buch wird mit dem autocite-Befehl
eingebunden \autocite{Tipler.2019}. Und das Literaturverzeichnis wird mit dem
nachfolgenden Befehl $\backslash$printbibliography gedruckt.
\printbibliography

Ein Inhaltsverzeichnis wird erstellt, indem Sie den folgenden Befehl
\textit{$\backslash$tableofcontents} direkt eine Zeile nach
\textit{$\backslash$begin\{document\}} einfügen.
% Sie können hierzu die bereits vorhandene Zeile mit \tableofcontents nach
% \begin{document} entkommentieren.

Auf dieselbe Art und Weise können Sie leicht ein Abbildungsverzeichnis mit
\textit{$\backslash$listoffigures} und ein Tabellenverzeichnisse mit
\textit{$\backslash$listoftables} erstellen. Binden Sie diese Verzeichnisse
direkt hier ein - es ist üblich diese Verzeichnisse am Ende der Arbeit
aufzuführen. Sollten Verzeichnisse nicht sofort erscheinen, kann es sein, dass
Sie einfach nochmals das Dokument kompilieren müssen.
%Sie können zur Einblendung von Abbildungs- und Tabellenverzeichnis die beiden nachfolgenden zeilen entkommentieren

\listoffigures

\listoftables

Weitere Befehle und Tips finden Sie im Anhang E der Vorlage im Repository \glqq
vorlage\_latex\_doku\grqq. In dieser Vorlage sind bereits viele weitere
Voreinstellungen vorgenommen und eine Grundstruktur vorgegeben, so dass Sie
sich von Anfang an auf den Inhalt der Arbeit konzentrieren können. Viel Spaß
beim Arbeiten mit \LaTeX!

\end{document}
